\documentclass[12pt]{article}
\usepackage[utf8]{inputenc}
\usepackage{cite}
\usepackage[spanish]{babel}
\selectlanguage{spanish}
\usepackage{graphicx}
\graphicspath{ {files/} }
\usepackage{url}
\usepackage{natbib}




\title{Reporte sobre la Actividad 2}
\author{García Parra Pedro}
\date{Febrero 2019}

\begin{document}

\maketitle

Para realizar esta actividad se requirió usar el lenguaje de progamación Python y un par de librerías llamadas: Pandas \cite{Pandas}, la cual es una librería 'open source' que sirve para el análisis de datos; la segunda librería fue matplotlib\cite{matplotlib}, ésta te proporciona herramientas para facilitar la realización un análisis visual de los datos, se pueden generar gáficas, histogramas, 'scatteplots', etc.

Para facilitar el desarrollo del código se utilizó Jupyter notebook (o Jupyter lab), éste es un programa que crea un localhost que se abre en un navegador web, aquí podemos usarlo como un editor de texto que nos permite ejecutar codigo en python de una manera muy sencilla y fácil de entender. En lo particular me gustó más Jupyter Lab porque éste tiene una intefaz más intuitiva una mayor facilidad para manejar diferentes archivos simultaneamente.

En esta actividad se requirió la elavoración de gráficas, y como ya se mencionó, se usó la librería matplotlib. Anteriormente, en otras materias, ya había creado gráficas en el lenguaje de programación Fortran con la ayuda de GNUPlot, y en general no hay una gran diferencia entre ambas maneras de crear gáficas; ambos pueden crear buenas gŕaficas. Entender la sintaxis de matplotlib fue algo complicado pero igual fue entender la sintaxis de GNUPlot. La gran diferencia entre los dos métodos es que al utilizar matplotlib no ocupas software adicional, todo se puede hacer desde dentro del lenguaje.

En general Python me pareció un muy buen lenguaje de programación con el cual hacer todo tipo de analisis de datos y todo lo que podría hacer Fortran de una manera más intuitiva; aunque en realidad no utilizaramos muchas funciones propoias de python sino utilizamos el de las dos librerías.

\bibliographystyle{plain}
\bibliography{M335}

\end{document}
